\section{Результаты}

\subsection{Тестирование программы}

Для проверки корректности работы программы был создан тестовый файл \texttt{test\_numbers.txt} со следующим содержимым:

\begin{verbatim}
15
20
8
12
7
\end{verbatim}

\subsection{Анализ тестовых данных}

\begin{table}[h!]
\centering
\begin{tabular}{|c|c|c|l|}
\hline
\textbf{Число} & \textbf{Простое?} & \textbf{Составное?} & \textbf{Действие} \\ \hline
15 & Нет & Да & Отправить родителю \\ \hline
20 & Нет & Да & Отправить родителю \\ \hline
8 & Нет & Да & Отправить родителю \\ \hline
12 & Нет & Да & Отправить родителю \\ \hline
7 & \textbf{Да} & Нет & \textbf{Завершить работу} \\ \hline
\end{tabular}
\caption{Анализ чисел из тестового файла}
\end{table}

\textbf{Ожидаемое поведение:}
\begin{itemize}
    \item Числа 15, 20, 8, 12 --- составные, должны быть отправлены родителю
    \item Число 7 --- простое, должно вызвать завершение обоих процессов
\end{itemize}

\subsection{Вывод программы}

\begin{verbatim}
Enter filename: test_numbers.txt
16-00-21MSG parent Reading file 'test_numbers.txt'
16-00-21MSG parent Sending number 15 to child
16-00-21MSG parent Received composite number 15 from child
16-00-21MSG parent Sending number 20 to child
16-00-21MSG parent Received composite number 20 from child
16-00-21MSG parent Sending number 8 to child
16-00-21MSG parent Received composite number 8 from child
16-00-21MSG parent Sending number 12 to child
16-00-21MSG parent Received composite number 12 from child
16-00-21MSG parent Sending number 7 to child
16-00-21MSG parent Received termination signal from child
16-00-21MSG parent Child process exited with code 0
16-00-21MSG parent Terminating
\end{verbatim}

\subsection{Проверка дополнительных случаев}

\textbf{Тест 1: Отрицательное число}

Файл: \texttt{-5}

Результат: программа корректно завершается при получении отрицательного числа.

\textbf{Тест 2: Только составные числа}

Файл: \texttt{4 6 8 9 10}

Результат: все числа обработаны и отправлены родителю, программа не завершается до конца файла.

\textbf{Тест 3: Простое число в начале}

Файл: \texttt{2 4 6}

Результат: программа завершается сразу после обработки числа 2 (простое).

\subsection{Проверка корректности работы с pipe}

Для проверки корректности межпроцессного взаимодействия было выполнено:

\begin{enumerate}
    \item \textbf{Проверка создания процесса:} Process Explorer показывает, что при запуске parent.exe создаётся дочерний процесс child.exe
    
    \item \textbf{Проверка дескрипторов:} Handle Count показывает корректное создание и закрытие pipe-дескрипторов
    
    \item \textbf{Проверка завершения:} При завершении child процесса parent корректно получает код возврата и завершается
\end{enumerate}

\subsection{Обработка ошибок}

\textbf{Тест: Несуществующий файл}

Ввод: \texttt{nonexistent.txt}

Результат:
\begin{verbatim}
Enter filename: nonexistent.txt
16-00-21ERR parent Cannot open file: nonexistent.txt
\end{verbatim}

Программа корректно обрабатывает ошибку и завершается с кодом 1.

\textbf{Тест: Отсутствие child.exe}

Если child.exe отсутствует в директории, parent выводит:
\begin{verbatim}
==================================================================
ASSERTION AT parent.cpp: 69
Condition: process_created
Message: Failed to create child process child.exe
==================================================================
\end{verbatim}

\subsection{Особенности реализации}

\begin{itemize}
    \item \textbf{Двунаправленный pipe:} Программа использует два pipe для полноценного обмена данными
    \item \textbf{Перенаправление stdin/stdout:} Child работает с pipe'ами через стандартные потоки
    \item \textbf{Правильное закрытие:} Все дескрипторы корректно закрываются, утечек ресурсов нет
    \item \textbf{Логирование с временными метками:} Позволяет отследить последовательность событий
\end{itemize}

\subsection{Выводы по результатам}

Программа работает корректно согласно варианту задания:
\begin{enumerate}
    \item Составные числа обрабатываются и отправляются родителю
    \item Простые и отрицательные числа вызывают завершение обоих процессов
    \item Межпроцессное взаимодействие через pipe реализовано корректно
    \item Все системные ошибки обрабатываются с выводом информативных сообщений
    \item Ресурсы корректно освобождаются при завершении
\end{enumerate}
