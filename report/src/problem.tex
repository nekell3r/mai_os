\section{Условие}
{\bfseries Цель работы:}

Приобретение практических навыков в:
\begin{itemize}
    \item Управлении процессами в ОС
    \item Обеспечении обмена данных между процессами посредством каналов
\end{itemize}

{\bfseries Задание:}

Составить и отладить программу на языке Си, осуществляющую работу с процессами и взаимодействие между ними в одной из двух операционных систем. В результате работы программа (основной процесс) должен создать для решения задачи один или несколько дочерних процессов. Взаимодействие между процессами осуществляется через системные сигналы/события и/или каналы (pipe). Необходимо обрабатывать системные ошибки, которые могут возникнуть в результате работы.

{\bfseries Вариант 10 (Группа вариантов 2):}

В файле записаны команды вида: <<число$<$endline$>$>>. Дочерний процесс производит проверку этого числа на простоту. Если число составное, то дочерний процесс пишет это число в стандартный поток вывода. Если число отрицательное или простое, то тогда дочерний и родительский процессы завершаются. Количество чисел может быть произвольным.

\textbf{Алгоритм работы:}
\begin{enumerate}
    \item Родительский процесс создаёт дочерний процесс
    \item Первой строкой пользователь в консоль родительского процесса вводит имя файла
    \item Файл будет использован для открытия файла с таким именем на чтение
    \item Стандартный поток ввода дочернего процесса переопределяется открытым файлом
    \item Дочерний процесс читает команды из стандартного потока ввода
    \item Стандартный поток вывода дочернего процесса перенаправляется в pipe1
    \item Родительский процесс читает из pipe1 и прочитанное выводит в свой стандартный поток вывода
    \item Родительский и дочерний процесс должны быть представлены разными программами
\end{enumerate}


