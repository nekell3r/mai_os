\section{Исходная программа}

Ниже представлен полный исходный код программы.

\subsection{Вспомогательные модули (common)}

\subsubsection{comm.h}

\lstinputlisting[language=C++,caption=comm.h,basicstyle=\tiny\ttfamily]{inc/comm.h}

\subsubsection{defines.h}

\lstinputlisting[language=C++,caption=defines.h,basicstyle=\tiny\ttfamily]{inc/defines.h}

\subsubsection{errors.h}

\lstinputlisting[language=C++,caption=errors.h,basicstyle=\tiny\ttfamily]{inc/errors.h}

\subsubsection{errors.hpp}

\lstinputlisting[language=C++,caption=errors.hpp,basicstyle=\tiny\ttfamily]{inc/errors.hpp}

\subsubsection{errors.cpp}

\lstinputlisting[language=C++,caption=errors.cpp,basicstyle=\tiny\ttfamily]{inc/errors.cpp}

\subsection{Основные модули}

\subsubsection{parent.cpp}

\lstinputlisting[language=C++,caption=parent.cpp,basicstyle=\small\ttfamily]{inc/parent.cpp}

\subsubsection{child.cpp}

\lstinputlisting[language=C++,caption=child.cpp,basicstyle=\small\ttfamily]{inc/child.cpp}

\subsection{Тестовый запуск программы}

\textbf{Команда запуска:}
\begin{verbatim}
D:\programming_projects\mai_os\lab1\build\src> echo test_numbers.txt | .\parent.exe
\end{verbatim}

\textbf{Вывод программы:}
\begin{verbatim}
Введите имя файла: 00-03-12MSG parent Читаю файл 'test_numbers.txt'
00-03-12MSG parent Отправляю число 15 дочернему процессу
00-03-12MSG parent Получено составное число 15 от дочернего процесса
00-03-12MSG parent Отправляю число 20 дочернему процессу
00-03-12MSG parent Получено составное число 20 от дочернего процесса
00-03-12MSG parent Отправляю число 8 дочернему процессу
00-03-12MSG parent Получено составное число 8 от дочернего процесса
00-03-12MSG parent Отправляю число 12 дочернему процессу
00-03-12MSG parent Получено составное число 12 от дочернего процесса
00-03-12MSG parent Отправляю число 7 дочернему процессу
00-03-12MSG parent Получен сигнал завершения от дочернего процесса
00-03-12MSG parent Дочерний процесс завершился с кодом 0
00-03-12MSG parent Завершение работы
\end{verbatim}

\subsection{Трассировка системных вызовов (WinDbg)}

Для трассировки системных вызовов Windows API использовался отладчик \textbf{WinDbg Preview}. Согласно документации\footnote{\url{http://windbg.info/doc/1-common-cmds.html}}, команды \texttt{bu} (breakpoint unresolved) позволяют устанавливать breakpoints на функции, которые ещё не загружены.

\subsubsection{Команды WinDbg}

\begin{lstlisting}[basicstyle=\tiny\ttfamily]
windbgx.exe parent.exe
.logopen trace.log
bu KERNELBASE!CreatePipe
bu KERNELBASE!CreateProcessW
bu KERNELBASE!WriteFile ".printf \"WriteFile(handle=%p)\\n\", @rcx; g"
bu KERNELBASE!ReadFile ".printf \"ReadFile(handle=%p)\\n\", @rcx; g"
g
\end{lstlisting}

\subsubsection{Результаты трассировки}

\textbf{Загрузка модулей и создание pipe'ов:}

\begin{verbatim}
ModLoad: ntdll.dll, KERNEL32.DLL, KERNELBASE.dll
ModLoad: advapi32.dll, msvcrt.dll, sechost.dll, RPCRT4.dll

Breakpoint 4 hit
KERNELBASE!CreatePipe:              <- Создание pipe_to_child
00007ffa`43574d80 488bc4  mov rax,rsp

Breakpoint 4 hit
KERNELBASE!CreatePipe:              <- Создание pipe_from_child
00007ffa`43574d80 488bc4  mov rax,rsp
\end{verbatim}

\textbf{Обмен данными (parent $\leftrightarrow$ child):}

Трассировка показывает циклический обмен данными через pipe. Дескрипторы:
\begin{itemize}
    \item \texttt{0x110} --- pipe для записи (parent $\rightarrow$ child)
    \item \texttt{0x114} --- pipe для чтения (child $\rightarrow$ parent)
    \item \texttt{0x6c} --- stdout (консоль для логов)
\end{itemize}

\begin{verbatim}
Цикл 1 (число 15):
WriteFile(handle=0x110)         <- Parent отправляет 15
ReadFile(handle=0x114)          <- Parent читает ответ 15

Цикл 2 (число 20):
WriteFile(handle=0x110)         <- Parent отправляет 20
ReadFile(handle=0x114)          <- Parent читает ответ 20

Цикл 3 (число 8):
WriteFile(handle=0x110)         <- Parent отправляет 8
ReadFile(handle=0x114)          <- Parent читает ответ 8

Цикл 4 (число 12):
WriteFile(handle=0x110)         <- Parent отправляет 12
ReadFile(handle=0x114)          <- Parent читает ответ 12

Цикл 5 (число 7):
WriteFile(handle=0x110)         <- Parent отправляет 7
ReadFile(handle=0x114)          <- Parent читает ответ -1 (завершение)

ntdll!NtTerminateProcess        <- Завершение процесса
\end{verbatim}

\subsubsection{Анализ трассировки}

\begin{enumerate}
    \item \textbf{Создание pipe'ов:} \texttt{CreatePipe} вызывается 2 раза для создания двунаправленного канала связи
    
    \item \textbf{Синхронный обмен:} Паттерн WriteFile $\rightarrow$ ReadFile повторяется 5 раз (по числу чисел в файле)
    
    \item \textbf{Обработка данных:} Child корректно определил составные числа (15, 20, 8, 12) и простое число (7)
    
    \item \textbf{Завершение:} При обнаружении простого числа child отправляет -1 и оба процесса завершаются
    
    \item \textbf{Всего передано:} 5 чисел $\times$ 4 байта $\times$ 2 направления = 40 байт данных
\end{enumerate}
