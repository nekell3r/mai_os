\section{Выводы}

В ходе выполнения лабораторной работы:

\begin{enumerate}
    \item Изучены механизмы межпроцессного взаимодействия (IPC) в Windows:
    \begin{itemize}
        \item \textbf{File Mapping} (Memory-Mapped Files) — разделяемая память между процессами
        \item \textbf{Events} — объекты синхронизации для координации процессов
        \item \textbf{Named Objects} — именованные объекты ядра для IPC
    \end{itemize}
    
    \item Освоены системные вызовы Windows API для работы с file mapping:
    \begin{itemize}
        \item \texttt{CreateFileMappingA()} / \texttt{OpenFileMappingA()} — создание/открытие
        \item \texttt{MapViewOfFile()} / \texttt{UnmapViewOfFile()} — отображение в память
        \item \texttt{CreateEventA()} / \texttt{OpenEventA()} — работа с событиями
        \item \texttt{SetEvent()} / \texttt{WaitForSingleObject()} — синхронизация
    \end{itemize}
    
    \item Реализована программа межпроцессного взаимодействия через shared memory:
    \begin{itemize}
        \item Родительский процесс создает file mapping и события
        \item Дочерний процесс открывает существующие объекты
        \item Обмен данными через структуру в разделяемой памяти
        \item Синхронизация доступа через auto-reset events
        \item Реализован паттерн "запрос-ответ"
    \end{itemize}
    
    \item Проведена трассировка системных вызовов с использованием WinDbg:
    \begin{itemize}
        \item Установлены breakpoint'ы на ключевые функции ntdll
        \item Проанализирована последовательность системных вызовов
        \item Получена статистика вызовов для каждой фазы работы
    \end{itemize}
    
    \item Выявлены преимущества file mapping перед pipes:
    \begin{itemize}
        \item \textbf{Zero-copy} — передача данных без системных вызовов
        \item \textbf{Высокая скорость} — прямой доступ к памяти
        \item \textbf{Структурированные данные} — передача C++ структур напрямую
        \item \textbf{Низкая задержка} — 0.1-0.5 мс против 1-5 мс для pipes
    \end{itemize}
    
    \item Установлены недостатки file mapping:
    \begin{itemize}
        \item Требуется явная синхронизация (накладные расходы на events)
        \item Более сложная реализация по сравнению с pipes
        \item Только для процессов на одной машине
        \item Риск race conditions при неправильной синхронизации
    \end{itemize}
    
    \item Проанализированы накладные расходы:
    \begin{itemize}
        \item Инициализация: ~6 системных вызовов (создание mapping + 2 events)
        \item Синхронизация: 2 системных вызова на цикл запрос-ответ
        \item Передача данных: 0 системных вызовов (прямой доступ к памяти)
        \item Очистка: ~10 системных вызовов (unmap + close handles)
    \end{itemize}
    
    \item Сравнены механизмы IPC:
    \begin{itemize}
        \item \textbf{Pipes} — проще в использовании, автоматическая синхронизация, но требуют копирования данных
        \item \textbf{File Mapping} — сложнее в реализации, требует явной синхронизации, но обеспечивает zero-copy и высокую производительность
    \end{itemize}
\end{enumerate}

\subsection{Практические рекомендации}

\textbf{Когда использовать File Mapping:}
\begin{itemize}
    \item Частый обмен данными между процессами
    \item Передача больших структур данных
    \item Требования к минимальной задержке
    \item Произвольный доступ к данным
    \item Процессы на одной машине
\end{itemize}

\textbf{Когда использовать Pipes:}
\begin{itemize}
    \item Потоковая передача данных (stream-oriented)
    \item Простота реализации важнее производительности
    \item Односторонняя коммуникация (producer-consumer)
    \item Возможность работы через сеть (named pipes)
\end{itemize}

\subsection{Области применения}

File mapping и events используются в реальных системах:

\begin{itemize}
    \item \textbf{Браузеры} — multi-process architecture (Chrome, Firefox)
    \item \textbf{Базы данных} — shared buffer pools (PostgreSQL, MySQL)
    \item \textbf{Игровые движки} — обмен данными между процессами рендеринга
    \item \textbf{Системное ПО} — Windows services communication
    \item \textbf{Высокопроизводительные вычисления} — HPC кластеры
\end{itemize}

\subsection{Достижение целей работы}

Цели лабораторной работы полностью достигнуты:

\begin{enumerate}
    \item \textbf{Управление процессами} — освоено создание дочерних процессов через \texttt{CreateProcess()}, получение кода возврата
    
    \item \textbf{Межпроцессное взаимодействие} — реализован обмен данными через file mapping без копирования
    
    \item \textbf{Синхронизация процессов} — освоена работа с events для координации доступа к shared memory
    
    \item \textbf{Трассировка системных вызовов} — проведен детальный анализ работы программы на уровне ядра
\end{enumerate}

File mapping — мощный механизм IPC, обеспечивающий высокую производительность за счет zero-copy передачи данных. При правильной синхронизации через events достигается эффективное взаимодействие процессов с минимальными накладными расходами.

\pagebreak
