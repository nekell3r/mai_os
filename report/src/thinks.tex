\section{Выводы}

В ходе выполнения лабораторной работы:

\begin{enumerate}
    \item Изучены методы работы с потоками в Windows API: \texttt{CreateThread()}, \texttt{WaitForMultipleObjects()}, \texttt{CloseHandle()}
    
    \item Реализован многопоточный медианный фильтр с разделением матрицы по строкам между потоками
    
    \item Проведено исследование производительности для матриц различных размеров (100×100, 500×500, 1000×1000) и количества потоков (1, 2, 4, 8)
    
    \item Получены следующие результаты для матрицы 1000×1000:
    \begin{itemize}
        \item Ускорение на 8 потоках: 4.86×
        \item Оптимальная конфигурация: 4 потока (эффективность 77\%)
        \item При 8 потоках эффективность падает до 61\% из-за накладных расходов
    \end{itemize}
    
    \item Установлено, что эффективность распараллеливания зависит от:
    \begin{itemize}
        \item Размера задачи (для больших матриц выше)
        \item Количества физических ядер процессора
        \item Накладных расходов на управление потоками
    \end{itemize}
    
    \item Освоены методы мониторинга потоков с помощью Task Manager и PowerShell
\end{enumerate}

Многопоточность эффективна для задач с независимыми вычислениями. Оптимальное количество потоков необходимо выбирать с учетом размера задачи и аппаратного обеспечения. Для процессора i7-10700K (8 ядер) оптимальным является использование 4 потоков.

Цели лабораторной работы достигнуты: получены практические навыки в управлении потоками в ОС Windows, реализован и исследован многопоточный алгоритм.

\pagebreak
