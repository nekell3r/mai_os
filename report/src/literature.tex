\section{Исходная программа}

Ниже представлен полный исходный код программы.

\subsection{Вспомогательные модули (common)}

\subsubsection{comm.h}

\lstinputlisting[language=C++,caption=comm.h,basicstyle=\tiny\ttfamily]{inc/comm.h}

\subsubsection{defines.h}

\lstinputlisting[language=C++,caption=defines.h,basicstyle=\tiny\ttfamily]{inc/defines.h}

\subsubsection{errors.h}

\lstinputlisting[language=C++,caption=errors.h,basicstyle=\tiny\ttfamily]{inc/errors.h}

\subsubsection{errors.hpp}

\lstinputlisting[language=C++,caption=errors.hpp,basicstyle=\tiny\ttfamily]{inc/errors.hpp}

\subsubsection{errors.cpp}

\lstinputlisting[language=C++,caption=errors.cpp,basicstyle=\tiny\ttfamily]{inc/errors.cpp}

\subsection{Основные модули}

\subsubsection{parent.cpp}

\lstinputlisting[language=C++,caption=parent.cpp,basicstyle=\small\ttfamily]{inc/parent.cpp}

\subsubsection{child.cpp}

\lstinputlisting[language=C++,caption=child.cpp,basicstyle=\small\ttfamily]{inc/child.cpp}

\subsection{Тестовый запуск программы}

\textbf{Команда запуска:}

\begin{verbatim}
D:\programming_projects\mai_os\lab1\build\src> echo test_numbers.txt | .\parent.exe
\end{verbatim}

\textbf{Реальный вывод программы:}

\begin{verbatim}
Введите имя файла: 19-18-31MSG parent Читаю файл 'test_numbers.txt'
19-18-31MSG parent Отправляю число 15 дочернему процессу
19-18-31MSG parent Получено составное число 15 от дочернего процесса
19-18-31MSG parent Отправляю число 20 дочернему процессу
19-18-31MSG parent Получено составное число 20 от дочернего процесса
19-18-31MSG parent Отправляю число 8 дочернему процессу
19-18-31MSG parent Получено составное число 8 от дочернего процесса
19-18-31MSG parent Отправляю число 12 дочернему процессу
19-18-31MSG parent Получено составное число 12 от дочернего процесса
19-18-31MSG parent Отправляю число 7 дочернему процессу
19-18-31MSG parent Получен сигнал завершения от дочернего процесса
19-18-31MSG parent Дочерний процесс завершился с кодом 0
19-18-31MSG parent Завершение работы
\end{verbatim}
