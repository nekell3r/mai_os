\section{Условие}
{\bfseries Цель работы:} приобретение практических навыков в:
\begin{itemize}
    \item Управление потоками в ОС
    \item Обеспечение синхронизации между потоками
\end{itemize}

{\bfseries Задание:} составить программу на языке Си, обрабатывающую данные в многопоточном режиме. При обработки использовать стандартные средства создания потоков операционной системы (Windows). Ограничение максимального количества потоков, работающих в один момент времени, должно быть задано ключом запуска программы.

Так же необходимо уметь продемонстрировать количество потоков, используемое программой с помощью стандартных средств операционной системы.

В отчете привести исследование зависимости ускорения и эффективности алгоритма от входных данных и количества потоков. Получившиеся результаты необходимо объяснить.

{\bfseries Вариант:} 11

Наложить K раз медианный фильтр на матрицу, состоящую из целых чисел. Размер окна задается пользователем.


