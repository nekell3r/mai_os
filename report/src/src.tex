\section{Метод решения}

\subsection{Общее описание алгоритма}

Программа реализует классическую схему межпроцессного взаимодействия типа <<producer-consumer>> на платформе Windows с использованием неименованных каналов (anonymous pipes).

\textbf{Архитектура решения:}

\begin{enumerate}
    \item \textbf{Родительский процесс (parent.exe):}
    \begin{itemize}
        \item Запрашивает имя файла с числами
        \item Создаёт два pipe для двусторонней связи с child
        \item Запускает дочерний процесс через \texttt{CreateProcess()}
        \item Перенаправляет stdin/stdout дочернего процесса на pipe'ы
        \item Читает числа из файла и отправляет их child'у
        \item Получает ответы от child'а и выводит их
        \item Завершает работу при получении сигнала -1
    \end{itemize}
    
    \item \textbf{Дочерний процесс (child.exe):}
    \begin{itemize}
        \item Читает числа из перенаправленного stdin
        \item Проверяет каждое число на простоту
        \item Если число составное --- отправляет его обратно
        \item Если число простое или отрицательное --- отправляет -1 и завершается
    \end{itemize}
\end{enumerate}

\textbf{Схема взаимодействия:}

\begin{verbatim}
User -> parent.exe -> pipe_to_child -> child.exe
                                          |
parent.exe <- pipe_from_child <-----------+
\end{verbatim}

\subsection{Алгоритм проверки простоты числа}

Используется классический алгоритм:
\begin{enumerate}
    \item Числа меньше 2 --- не простые
    \item Число 2 --- простое
    \item Чётные числа --- не простые
    \item Для нечётных проверяем делимость на все нечётные числа от 3 до $\sqrt{n}$
\end{enumerate}

Сложность: $O(\sqrt{n})$

\section{Описание программы}

\subsection{Структура проекта}

Проект организован по модульному принципу:

\begin{itemize}
    \item \texttt{common/} --- вспомогательные модули (логирование, обработка ошибок)
    \item \texttt{src/parent.cpp} --- родительский процесс
    \item \texttt{src/child.cpp} --- дочерний процесс
\end{itemize}

\subsection{Основные типы данных}

\begin{itemize}
    \item \texttt{HANDLE} --- дескриптор Windows-объекта (pipe, процесс, поток)
    \item \texttt{SECURITY\_ATTRIBUTES} --- атрибуты безопасности для pipe
    \item \texttt{STARTUPINFO} --- информация для запуска процесса (перенаправление stdin/stdout)
    \item \texttt{PROCESS\_INFORMATION} --- информация о созданном процессе
\end{itemize}

\subsection{Основные функции}

\textbf{В parent.cpp:}
\begin{itemize}
    \item \texttt{main()} --- основная функция родительского процесса
    \begin{itemize}
        \item Создаёт pipe'ы через \texttt{CreatePipe()}
        \item Настраивает наследование дескрипторов через \texttt{SetHandleInformation()}
        \item Запускает child через \texttt{CreateProcess()}
        \item Организует обмен данными через \texttt{ReadFile()} и \texttt{WriteFile()}
    \end{itemize}
\end{itemize}

\textbf{В child.cpp:}
\begin{itemize}
    \item \texttt{is\_prime(int n)} --- проверка числа на простоту
    \item \texttt{main()} --- основная функция дочернего процесса
    \begin{itemize}
        \item Получает дескрипторы stdin/stdout через \texttt{GetStdHandle()}
        \item Читает числа из stdin
        \item Проверяет условия завершения
        \item Отправляет результаты в stdout
    \end{itemize}
\end{itemize}

\subsection{Системные вызовы Windows API}

\begin{enumerate}
    \item \texttt{CreatePipe()} --- создание неименованного канала (pipe)
    \begin{itemize}
        \item Параметры: указатели на read/write дескрипторы, атрибуты безопасности, размер буфера
        \item Возвращает: TRUE при успехе
    \end{itemize}
    
    \item \texttt{SetHandleInformation()} --- управление наследованием дескрипторов
    \begin{itemize}
        \item Позволяет явно указать, какие дескрипторы наследуются дочерним процессом
        \item Используется флаг \texttt{HANDLE\_FLAG\_INHERIT}
    \end{itemize}
    
    \item \texttt{CreateProcess()} --- создание нового процесса
    \begin{itemize}
        \item Параметры: путь к exe, аргументы, атрибуты безопасности, флаги, окружение, рабочая директория, STARTUPINFO, PROCESS\_INFORMATION
        \item Позволяет перенаправить stdin/stdout/stderr через \texttt{STARTUPINFO.hStdInput/hStdOutput/hStdError}
    \end{itemize}
    
    \item \texttt{GetStdHandle()} --- получение стандартного дескриптора
    \begin{itemize}
        \item Параметры: тип дескриптора (STD\_INPUT\_HANDLE, STD\_OUTPUT\_HANDLE, STD\_ERROR\_HANDLE)
        \item Используется в child для получения перенаправленных дескрипторов
    \end{itemize}
    
    \item \texttt{ReadFile()} --- чтение данных из дескриптора
    \begin{itemize}
        \item Параметры: дескриптор, буфер, размер, указатель на количество прочитанных байт
        \item Используется для чтения из pipe
    \end{itemize}
    
    \item \texttt{WriteFile()} --- запись данных в дескриптор
    \begin{itemize}
        \item Параметры: дескриптор, буфер, размер, указатель на количество записанных байт
        \item Используется для записи в pipe
    \end{itemize}
    
    \item \texttt{WaitForSingleObject()} --- ожидание завершения процесса
    \begin{itemize}
        \item Параметры: дескриптор процесса, timeout (INFINITE для бесконечного ожидания)
        \item Блокирует выполнение до завершения дочернего процесса
    \end{itemize}
    
    \item \texttt{GetExitCodeProcess()} --- получение кода возврата процесса
    \begin{itemize}
        \item Параметры: дескриптор процесса, указатель на переменную для кода
    \end{itemize}
    
    \item \texttt{CloseHandle()} --- закрытие дескриптора
    \begin{itemize}
        \item Освобождает ресурсы, связанные с дескриптором
        \item Критически важно для корректного завершения pipe'ов
    \end{itemize}
\end{enumerate}

\subsection{Особенности реализации}

\textbf{1. Правильное наследование дескрипторов:}

По умолчанию создаём pipe'ы с \texttt{bInheritHandle = FALSE}, затем явно указываем, какие дескрипторы должны наследоваться через \texttt{SetHandleInformation()}. Это предотвращает утечки ресурсов.

\textbf{2. Перенаправление потоков:}

Child процесс получает перенаправленные stdin/stdout через \texttt{STARTUPINFO}, что позволяет ему работать с pipe'ами как с обычными потоками ввода-вывода.

\textbf{3. Закрытие ненужных концов pipe'ов:}

Parent закрывает те концы pipe'ов, которые использует child, чтобы избежать deadlock'ов. Child автоматически получает только нужные ему дескрипторы.

\textbf{4. Обработка ошибок:}

Все системные вызовы проверяются через макрос \texttt{ASSERT\_MSG()}, который выводит подробную информацию об ошибке (файл, строка, условие) и завершает программу через \texttt{exit(1)}.

\textbf{5. Логирование:}

Используются макросы \texttt{LogMsg()} и \texttt{LogErr()} с временными метками в формате HH-MM-SS, что позволяет отслеживать последовательность событий.
