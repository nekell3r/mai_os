\section{Условие}
{\bfseries Цель работы:} приобретение практических навыков в:
\begin{itemize}
    \item Управление процессами в ОС
    \item Организация межпроцессного взаимодействия (IPC)
    \item Синхронизация процессов
\end{itemize}

{\bfseries Задание:} составить программу на языке С++, осуществляющую межпроцессное взаимодействие между родительским и дочерним процессами через разделяемую память (file mapping) на Windows. Для синхронизации использовать события (Events).

Родительский процесс читает данные из файла и передает их дочернему процессу через разделяемую память. Дочерний процесс обрабатывает данные и возвращает результат обратно родителю.

В отчете привести исследование работы программы с использованием трассировки системных вызовов (WinDbg). Описать механизмы file mapping, events и их взаимодействие.

{\bfseries Вариант:} 10

Файл содержит числа. Дочерний процесс проверяет каждое число:
\begin{itemize}
    \item Если число \textbf{отрицательное} — завершает работу (родитель тоже завершается)
    \item Если число \textbf{простое} — завершает работу (родитель тоже завершается)
    \item Если число \textbf{составное} — отправляет его обратно родителю
\end{itemize}

