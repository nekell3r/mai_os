\section{Выводы}

В ходе выполнения лабораторной работы:

\begin{enumerate}
    \item Изучены механизмы работы с динамическими библиотеками в Windows:
    \begin{itemize}
        \item \textbf{Статическая линковка} — включение кода библиотеки в исполняемый файл на этапе компиляции
        \item \textbf{Динамическая загрузка} — загрузка DLL во время выполнения программы
        \item \textbf{Системные вызовы} — \texttt{LoadLibraryA()}, \texttt{GetProcAddress()}, \texttt{FreeLibrary()}
    \end{itemize}
    
    \item Освоены системные вызовы Windows API для работы с DLL:
    \begin{itemize}
        \item \texttt{LoadLibraryA()} / \texttt{LdrLoadDll()} — загрузка DLL в память
        \item \texttt{GetProcAddress()} / \texttt{LdrGetProcedureAddress()} — получение адреса функции
        \item \texttt{FreeLibrary()} / \texttt{LdrUnloadDll()} — выгрузка DLL из памяти
        \item \texttt{NtMapViewOfSection()} / \texttt{NtUnmapViewOfSection()} — маппинг DLL в адресное пространство
    \end{itemize}
    
    \item Реализованы две программы для демонстрации разных подходов:
    \begin{itemize}
        \item \textbf{Program 1} — статическая линковка, код сортировки включен в .exe
        \item \textbf{Program 2} — динамическая загрузка, переключение между реализациями во время выполнения
    \end{itemize}
    
    \item Созданы динамические библиотеки с единым контрактом:
    \begin{itemize}
        \item \texttt{bubble\_sort.dll} — реализация пузырьковой сортировки ($O(n^2)$)
        \item \texttt{quicksort.dll} — реализация сортировки Хоара ($O(n \log n)$)
        \item Обе библиотеки экспортируют функцию \texttt{Sort()} с одинаковой сигнатурой
    \end{itemize}
    
    \item Проведена трассировка системных вызовов с использованием WinDbg:
    \begin{itemize}
        \item Установлены breakpoint'ы на ключевые функции ntdll
        \item Проанализирована последовательность системных вызовов при загрузке DLL
        \item Выявлены различия между статической и динамической линковкой
    \end{itemize}
    
    \item Выявлены преимущества динамических библиотек:
    \begin{itemize}
        \item \textbf{Модульность} — можно обновлять библиотеки без перекомпиляции программы
        \item \textbf{Экономия памяти} — одна копия DLL может использоваться несколькими процессами
        \item \textbf{Гибкость} — выбор реализации во время выполнения
        \item \textbf{Разделение кода} — уменьшение размера исполняемых файлов
    \end{itemize}
    
    \item Установлены недостатки динамических библиотек:
    \begin{itemize}
        \item Требуется наличие DLL файлов в системе
        \item Накладные расходы на загрузку во время выполнения
        \item Возможны проблемы с версиями (DLL Hell)
        \item Больше системных вызовов при запуске программы
    \end{itemize}
    
    \item Проанализированы накладные расходы:
    \begin{itemize}
        \item \textbf{Статическая линковка:} 0 системных вызовов на этапе выполнения (все разрешено при компиляции)
        \item \textbf{Динамическая загрузка:} ~4 системных вызова на загрузку DLL (\texttt{LdrLoadDll}, \texttt{NtOpenSection}, \texttt{NtMapViewOfSection}, \texttt{LdrGetProcedureAddress})
        \item \textbf{Переключение DLL:} ~6 системных вызовов (выгрузка + загрузка)
        \item \textbf{Выполнение кода:} 0 системных вызовов (код выполняется напрямую в адресном пространстве)
    \end{itemize}
    
    \item Сравнены подходы к использованию библиотек:
    \begin{itemize}
        \item \textbf{Статическая линковка} — проще в использовании, нет зависимости от DLL, но больший размер .exe и невозможность изменения без перекомпиляции
        \item \textbf{Динамическая загрузка} — гибче, меньше размер .exe, возможность переключения реализаций, но требует наличия DLL и больше системных вызовов
    \end{itemize}
\end{enumerate}

\subsection{Практические рекомендации}

\textbf{Когда использовать статическую линковку:}
\begin{itemize}
    \item Простые программы с небольшими библиотеками
    \item Когда важна независимость от внешних файлов
    \item Когда размер .exe не критичен
    \item Когда реализация не будет меняться
\end{itemize}

\textbf{Когда использовать динамическую загрузку:}
\begin{itemize}
    \item Большие библиотеки, используемые несколькими процессами
    \item Когда нужна возможность обновления без перекомпиляции
    \item Когда нужна гибкость выбора реализации во время выполнения
    \item Когда важна модульность архитектуры
    \item Плагины и расширения
\end{itemize}

\subsection{Области применения}

Динамические библиотеки широко используются в реальных системах:

\begin{itemize}
    \item \textbf{Операционные системы} — системные DLL (kernel32.dll, user32.dll, ntdll.dll)
    \item \textbf{Браузеры} — плагины и расширения (Flash, PDF viewers)
    \item \textbf{Игровые движки} — загрузка ресурсов и модов
    \item \textbf{Базы данных} — драйверы и расширения (PostgreSQL, MySQL)
    \item \textbf{Графические редакторы} — фильтры и плагины (Photoshop, GIMP)
    \item \textbf{IDE и редакторы} — плагины и расширения (Visual Studio, VS Code)
\end{itemize}

\subsection{Достижение целей работы}

Цели лабораторной работы полностью достигнуты:

\begin{enumerate}
    \item \textbf{Создание динамических библиотек} — освоено создание DLL с экспортируемыми функциями, использование \texttt{extern "C"} для предотвращения name mangling
    
    \item \textbf{Статическая линковка} — реализована программа, использующая библиотеку на этапе компиляции
    
    \item \textbf{Динамическая загрузка} — реализована программа, загружающая DLL во время выполнения с возможностью переключения между реализациями
    
    \item \textbf{Анализ подходов} — проведено сравнение статической и динамической линковки, выявлены преимущества и недостатки каждого подхода
    
    \item \textbf{Трассировка системных вызовов} — проведен детальный анализ работы программы на уровне ядра, выявлена последовательность системных вызовов при загрузке DLL
\end{enumerate}

Динамические библиотеки — мощный механизм модульности в Windows, обеспечивающий гибкость и возможность обновления кода без перекомпиляции основной программы. При правильном использовании они позволяют создавать расширяемые и поддерживаемые системы.

\pagebreak
