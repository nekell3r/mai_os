\section{Выводы}

В ходе выполнения лабораторной работы были приобретены практические навыки в управлении процессами и организации межпроцессного взаимодействия на платформе Windows.

\textbf{Основные достижения:}

\begin{enumerate}
    \item Изучены основные механизмы создания процессов в Windows через API функцию \texttt{CreateProcess()}
    
    \item Освоена работа с неименованными каналами (anonymous pipes) для организации двунаправленного обмена данными между процессами
    
    \item Изучены механизмы перенаправления стандартных потоков ввода-вывода дочернего процесса через структуру \texttt{STARTUPINFO}
    
    \item Освоены методы управления наследованием дескрипторов через \texttt{SetHandleInformation()}, что критически важно для предотвращения утечек ресурсов и deadlock'ов
    
    \item Реализована корректная обработка системных ошибок с использованием макросов для информативного вывода диагностики
\end{enumerate}

\textbf{Практическая значимость:}

Межпроцессное взаимодействие через pipe является фундаментальным механизмом современных операционных систем. Полученные навыки применимы для разработки:
\begin{itemize}
    \item Распределённых систем обработки данных
    \item Систем с разделением привилегий (когда критичные операции выполняются в отдельном процессе)
    \item Систем мониторинга и управления процессами
    \item Интеграции различных компонентов через стандартные потоки ввода-вывода
\end{itemize}

\textbf{Важные практические навыки:}

\begin{itemize}
    \item Понимание жизненного цикла процесса в Windows
    \item Умение работать с дескрипторами и управлять их наследованием
    \item Знание особенностей синхронизации при работе с pipe'ами
    \item Навыки отладки межпроцессного взаимодействия
\end{itemize}

\pagebreak
