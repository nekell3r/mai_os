\section{Результаты}

\subsection{Тестовая конфигурация}

Тестирование проводилось на следующей конфигурации:
\begin{itemize}
    \item Процессор: Intel Core i7-10700K (8 ядер, 16 потоков)
    \item ОС: Windows 11
    \item Компилятор: MSVC, оптимизация Release (/O2)
\end{itemize}

\subsection{Параметры тестирования}

\begin{itemize}
    \item Размеры матриц: 100×100, 500×500, 1000×1000
    \item Размер окна фильтра: 5×5, количество итераций (K): 3
    \item Количество потоков: 1, 2, 4, 8
    \item Каждый тест повторялся 3 раза для усреднения
\end{itemize}

Метрики: ускорение $S(N) = T(1)/T(N)$, эффективность $E(N) = S(N)/N$.

\subsection{Результаты измерений}

\begin{table}[h]
\centering
\begin{tabular}{|c|c|c|c|}
\hline
Потоки & Время (мс) & Ускорение & Эффективность \\
\hline
\multicolumn{4}{|c|}{Матрица 100×100} \\
\hline
1 & 63.25 & 1.00 & 100\% \\
2 & 38.75 & 1.63 & 82\% \\
4 & 23.26 & 2.72 & 68\% \\
8 & 15.96 & 3.96 & 50\% \\
\hline
\multicolumn{4}{|c|}{Матрица 500×500} \\
\hline
1 & 1495.27 & 1.00 & 100\% \\
2 & 829.68 & 1.80 & 90\% \\
4 & 495.66 & 3.02 & 75\% \\
8 & 305.20 & 4.90 & 61\% \\
\hline
\multicolumn{4}{|c|}{Матрица 1000×1000} \\
\hline
1 & 6238.52 & 1.00 & 100\% \\
2 & 3316.28 & 1.88 & 94\% \\
4 & 2015.70 & 3.09 & 77\% \\
8 & 1282.90 & 4.86 & 61\% \\
\hline
\end{tabular}
\caption{Результаты тестирования производительности}
\end{table}

\subsection{Графики}

\begin{figure}[h]
\centering
\includegraphics[width=0.8\textwidth]{img/speedup.png}
\caption{Зависимость ускорения от количества потоков}
\end{figure}

\begin{figure}[h]
\centering
\includegraphics[width=0.8\textwidth]{img/efficiency.png}
\caption{Зависимость эффективности от количества потоков}
\end{figure}

\subsection{Анализ результатов}

Основные наблюдения:

\begin{enumerate}
    \item При использовании 2 потоков ускорение составляет 1.6--1.9× (80--95\% от идеального)
    
    \item При 4 потоках ускорение 2.7--3.1× (68--77\% от идеального) — оптимальная конфигурация
    
    \item При 8 потоках ускорение 4.0--4.9× (50--61\% от идеального) — эффективность падает из-за накладных расходов
    
    \item Эффективность выше для больших матриц, так как время вычислений превышает накладные расходы на управление потоками
\end{enumerate}

Эффективность снижается с ростом числа потоков из-за:
\begin{itemize}
    \item Накладных расходов на создание и синхронизацию потоков
    \item Конкуренции за кэш процессора и пропускную способность памяти
    \item Переключения контекста при превышении числа физических ядер
\end{itemize}

Для процессора i7-10700K (8 ядер) оптимальным является использование 4 потоков, обеспечивающее эффективность 75--77\%.

Реальное ускорение на 8 потоках (4.86×) близко к теоретическому максимуму по закону Амдала, что говорит о высокой степени параллелизуемости алгоритма (95--97\%).
