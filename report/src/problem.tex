\section{Условие}
{\bfseries Цель работы:} приобретение практических навыков в:
\begin{itemize}
    \item Создание динамических библиотек
    \item Создание программ, которые используют функции динамических библиотек
\end{itemize}

{\bfseries Задание:} требуется создать динамические библиотеки, которые реализуют заданный вариантом функционал. Далее использовать данные библиотеки 2-мя способами:

\begin{enumerate}
    \item Во время компиляции (на этапе «линковки»/linking)
    \item Во время исполнения программы. Библиотеки загружаются в память с помощью интерфейса ОС для работы с динамическими библиотеками
\end{enumerate}

В конечном итоге, в лабораторной работе необходимо получить следующие части:
\begin{itemize}
    \item Динамические библиотеки, реализующие контракты, которые заданы вариантом
    \item Тестовая программа (программа №1), которая использует одну из библиотек, используя информацию полученные на этапе компиляции
    \item Тестовая программа (программа №2), которая загружает библиотеки, используя только их относительные пути и контракты
\end{itemize}

Провести анализ двух типов использования библиотек.

Пользовательский ввод для обоих программ должен быть организован следующим образом:
\begin{enumerate}
    \item Если пользователь вводит команду «0», то программа переключает одну реализацию контрактов на другую (необходимо только для программы №2). Можно реализовать лабораторную работу без данной функции, но максимальная оценка в этом случае будет «хорошо»
    \item «1 arg1 arg2 … argN», где после «1» идут аргументы для первой функции, предусмотренной контрактами. После ввода команды происходит вызов первой функции, и на экране появляется результат её выполнения
    \item «2 arg1 arg2 … argM», где после «2» идут аргументы для второй функции, предусмотренной контрактами. После ввода команды происходит вызов второй функции, и на экране появляется результат её выполнения
\end{enumerate}

В отчете привести исследование работы программы с использованием трассировки системных вызовов (WinDbg). Описать механизмы загрузки динамических библиотек и их использование.

{\bfseries Вариант:} 9

Отсортировать целочисленный массив:
\begin{itemize}
    \item \textbf{Метод 1:} Пузырьковая сортировка
    \item \textbf{Метод 2:} Сортировка Хоара (QuickSort)
\end{itemize}

Контракт функции: \texttt{Int * Sort(int * array)}

